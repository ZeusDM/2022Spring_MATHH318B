\lecture{3}{2022-01-24}{}

\section{Interaction with Boundness}

\begin{prop}{Pointwise convergence does not preserve boundness}{}
  There exists a sequence of functions \(f_n \colon A \subset \mathbb{R} \to \mathbb{R}\), all of them bounded on \(A\), and \(f_n \to f\) pointwise on \(A\) for a unbounded function \(f\) on \(A\).
\end{prop}

\begin{dem}{}{}
  Consider \(f_n \colon \mathbb{R} \to \mathbb{R}\) defined by \[
    f_n(x) = 
	\begin{cases}
	  x & |x| < n \\
	  0 & \text{otherwise,}
	\end{cases}
  \]
  which converges pointwise to \(f(x) = x\) on \(\mathbb{R}\).
\end{dem}

\begin{prop}{Uniform convergence preserves boundness}{}
  If \(f_n \colon A \subset \mathbb{R} \to \mathbb{R}\) is bounded for each \(n\), and if \(f_n \to f\) uniformly on \(A\), then \(f\) is bounded on \(A\).
\end{prop}

\begin{dem}{}{}
  Plug \(\epsilon \mapsto 1\) on the definition of uniformly convergence. Then, there exists \(N\) such that \[
    |f_n(x) - f(x)| < 1
  \] 
  for all \(n \geq N\) and all \(x \in A\).

  Since \(f_N\) is bounded, there exists \(M\) such that \(|f_N(x)| < M\) for all \(x \in A\).
  Finally, by triangular inequality, we conclude that \[
    |f(x)| < M + 1,
  \] 
  for all \(x \in A\); therefore, \(f\) is bounded.
\end{dem}

\section{Interaction with Continuity}

\begin{prop}{Uniform convergence preserves continuity}{}
  If \(f_n \colon A \subset \mathbb{R} \to \mathbb{R}\) is continuous for each \(n\), and if \(f_n \to f\) uniformly on \(A\), then \(f\) is continuous on \(A\).
\end{prop}

\begin{dem}{}{}
  Let \(c \in A\) be arbitrary.
  Let \(\epsilon > 0\) be arbitrary.

  Since \(f_n \to f\), there exists \(N\) such that \[
    |f_n(x) - f(x)| < \epsilon/3
  \] for all \(n \geq N\) and \(x \in A\).

  Since \(f_N\) is continous at \(c\), there exists \(\delta > 0\) such that \[
    |f_N(x) - f_N(c)| < \epsilon/3
  \]  for all \(x \in A\) satisfying \(|x - c| < \delta\).

  Therefore, by triangle inequality, 
  \begin{align*}
    |f(c) - f(x)| &= |f(c) - f_n(c) + f_n(c) - f_n(x) + f_n(x) - f(x)| \\
                  &\leq |f(c) - f_n(c)| + |f_n(c) - f_n(x)| + |f_n(x) - f(x)| \\
                  &< \epsilon
  \end{align*}
  for all \(x \in A\) satisfying \(|x - c| < \delta\).
  Since \(\epsilon\) was arbitrary, this implies \(f\) is continuous at \(c\).
  Since \(c\) was arbitrary, this implies \(f\) is continuous on \(A\).
\end{dem}

\section{Interaction with Differentiability}

\begin{prop}{Uniform convergence does not preserve differentiability}{}
  There exists a sequence of functions \(f_n \colon A \subset \mathbb{R} \to \mathbb{R}\), all of them differentiable on \(A\), and \(f_n \to f\) uniformly on \(A\) for a non-differentiable function \(f\) on \(A\).
\end{prop}

\begin{thm}{}{}
  If
  \begin{enumerate}
    \item \(f_n\) is differentiable on \([a, b]\), for all integers \(n\),
    \item \(f_n'\) converges uniformly on \([a, b]\) to \(g\), and
    \item \(f_n\) converges pointwise on \([a, b]\) to \(f\),
  \end{enumerate}
  then \(f\) is differentiable on \([a, b]\), and  \(f'=g\).
\end{thm}
