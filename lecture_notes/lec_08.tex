\lecture{8}{2022-02-04}{}

\section{Measure}

We are going to cheat a little.

\begin{defn}{Measure Zero}{}
  A set \(A \subset \mathbb{R}\) \emph{has measure zero} if, for all \(\epsilon > 0\), there exists a finite or countable collection of open intervals \((a_n, b_n)\) such that \[
    A \subset \bigcup_{n} (a_n, b_n) \qquad \text{and} \qquad \sum_{n} (b_n - a_n) \leq \epsilon.
  \] 
\end{defn}

\begin{prop}{}{}
  If \(A\) has measure zero, and \(B\) has measure zero, then \(A \cup B\) has measure zero.
\end{prop}

\begin{prop}{}{}
  If \(\mathcal{C}\) is a countable collection of sets with measure zero, then their union also has measure zero.
\end{prop}

\begin{exmp}{}{}
  The sets \(\{4, 8\}\), \(\mathbb{N}\), \(\mathbb{Q}\), and the cantor set \(C\) have measure zero.
\end{exmp}

\begin{prop}{}{}
  If \(a < b\), then \([a, b]\) does not have measure zero.
\end{prop}

\begin{dem}{Sketch}{}
  Suppose we could cover \([a, b]\) with an open cover with total length at most \(\frac{b-a}{4}\).
  Since \([a, b]\) is compact, this cover has a finite sub-cover; which still has length at most \(\frac{b-a}{4}\).
  Suppose that that subcover contains \((a_1, b_1)\), \(\dots\), \((a_n, b_n)\), with \(a_1 \leq a_2 \leq \cdots \leq a_n\).
  \textcolor{red}{To be finished.}
\end{dem}

\begin{cor}{}{}
  If \(a < b\), then \((a, b]\), \([a, b)\), \((a, b)\) do not have measure zero.
\end{cor}

\subsection{Step Functions}

A step function is a function that is non-zero on a finite set of disjoint bounded intervals, constant on each of those intervals, and zero everywhere else.

\begin{defn}{Characteristic Function}{}
  Given \(S \subset \mathbb{R}\), the corresponding characteristic function is \[
    \chi_S(x) =
    \begin{cases}
      1 & x \in S \\
      0 & x \notin S
    \end{cases}
  \] 
\end{defn}

\begin{defn}{Step Function}{}
  A step function is a function of the form \[
    f(x) = c_1\chi_{I_1}(x) + c_2\chi_{I_2}(x) + \cdots + c_n\chi_{I_n}(x),
  \] 
  where \(c_j\) are real numbers and \(I_j\) is a collection of disjoint bounded intervals.
  Equivalently, \(I_j\) can be a collection of (not necessarily disjoint) bounded intervals.
\end{defn}

\begin{defn}{Lebesgue integral of a step function}{}
  Given a step function \[
    f(x) = c_1\chi_{I_1}(x) + c_2\chi_{I_2}(x) + \cdots + c_n\chi_{I_n}(x),
  \] we define its Lebesgue integral to be \[
    \int_{-\infty}^\infty f(x) = c_1m(I_1) + c_2m(I_2) + \cdots + c_nm(I_n),
  \] 
  where \(m(I_i)\) is the absolute value of the difference of \(I_i\)'s endpoints.
\end{defn}

\begin{defn}{Lebesgue integral of some other functions}{}
  Given a function \(f\colon \mathbb{R} \to \mathbb{R}\), if we can find a sequence \(\phi_n\) of step functions such that
  \begin{enumerate}
    \item \(\phi_1(x), \phi_2(x), \phi_3(x), \dots\) is monotone increasing for each \(x \in \mathbb{R}\), and
    \item \(\phi_n \to f\) pointwise except possibly on a set of measure zero,
  \end{enumerate}
  then we define \[
    \int_{-\infty}^\infty f(x)dx = \lim_{n\to\infty} \int_{-\infty}^\infty \phi_n(x)dx,
  \] 
  if that limit exists.
\end{defn}
