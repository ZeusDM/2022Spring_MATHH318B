\lecture{2}{2022-01-21}{Examples of Uniform Convergence}

Uniform convergence of \(f_n\) to \(f\) says that, for all \(\epsilon > 0\), there exists \(N\) large enough so that the graph of \(f_n\), for all \(n \geq N\), is entirely in the ``\(\epsilon\)-tube'' of the graph of \(f\).

\begin{figure}[htbp]
  \centering
  \asyinclude{figures/asy/e-tube.asy}
  \caption{Graph of the ``\(\epsilon\)-tube.'' In this example, all \(f_n\), for \(n \geq 4\), are in the  \(\epsilon\)-tube.}
\end{figure}

\begin{exmp}{}{}
  Let \(f_n(x) = \frac{1}{1+nx^2}\) and let \(f\) be its poitwise limit, i.e., \(f(x) = 0\). Then, \(f_n\) uniformly converges to \(f\) on \((\epsilon, 1)\), for all \(\epsilon > 0\); however, \(f_n\) does not converge uniformly to \(f\) on \((0, 1)\).
\end{exmp}

\begin{figure}[htbp]
  \centering
  \asyinclude{figures/asy/frac11+nx2.asy}
  \caption{Graph of some functions \(f_n(x) = \frac{1}{1 + nx^2}\).}
\end{figure}
