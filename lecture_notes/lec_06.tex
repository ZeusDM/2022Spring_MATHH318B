\lecture{6}{2022-01-31}{}

\section{Uniformly Cauchy}

\begin{defn}{Uniformly Cauchy}{}
  The sequence of functions \(f_n \colon A \to \mathbb{R}\) is uniformly Cauchy on \(A\) if, for all \(\epsilon > 0\), there exists a positive integer \(N\) such that, for all \(x \in A\) and all \(m, n \geq N\),
  \[
    |f_m(x) - f_n(x)| < \epsilon.
  \] 
\end{defn}

\begin{thm}{}{}
  The sequence of functions \(f_n \colon A \to \mathbb{R}\) converges uniformly on \(A\) if, and only if, it is uniformly Cauchy on \(A\).
\end{thm}

\begin{dem}{Uniformly Convergence implies Uniformly Cauchy}{}
  If \(f_n \colon A \to \mathbb{R}\) converges uniformly on \(A\), then for all  \(\epsilon > 0\), there exists a positive integer  \(N\) such that \[
    |f_n(x) - f(x)| < \epsilon/2
  \] 
  for all \(x \in A\) and all \(n \geq N\).

  Therefore, it follows that, for all \(\epsilon\), there exists a positive integer \(N\) such that \[
    |f_m(x) - f_n(x)| = |f_m(x) - f(x)| + |f(x) - f_n(x)| < \epsilon.
  \] 
  for all \(x \in A\) and all \(m, n \geq N\).
\end{dem}

\begin{dem}{Uniformly Cauchy implies Uniformly Convergence}{}
  \textcolor{red}{To be done.}
\end{dem}

\section{Weierstrass M-test}

\begin{thm}{Weierstrass M-test}{wmtest}
  If \(g_n \colon A \to \mathbb{R}\) is a sequence of functions and if there exists constants \(M_n \geq 0\) so that \[
    |g_n(x)| \leq M_n
  \] 
  for all \(x \in A\)
  and \(\sum_{n} M_n\) converges, then \(\sum_{n} g_n(x)\) converges uniformly on \(A\).
\end{thm}

\begin{dem}{}{}
  Since \(\sum M_n\) converges, by the Cauchy criterion, for all \(\epsilon > 0\), there exists a positive integer \(N\) such that \[
	M_{m+1} + \cdots + M_n = |M_{m+1} + \cdots + M_n| < \epsilon
  \]
  for all \(n > m \geq N\).

  Thus, for all \(\epsilon\), for the  \(N\) above, implies that \[
    |g_{m+1}(x) + \cdots + g_n(x)| < \epsilon.
  \] 
  for all \(m > n \geq N\) and all \(x \in A\).
\end{dem}

\chapter{Function Spaces}

\section{Our first function spaces}

\begin{defn}{}{}
  Given \(A \subset \mathbb{R}\), let \(C(A)\) be the set of all functions that are continuous on  \(A\), and let \(B(A)\) be the set of all functions that are bounded on \(A\).
\end{defn}

\begin{prop}{}{}
  \(C([a, b]) \subseteq B([a, b]).\)
\end{prop}

\begin{defn}{Infinity Norm}{}
  Given a bounded function \(f\colon [a, b] \to \mathbb{R}\), let \[
    ||f||_\infty = \sup_{x \in [a, b]} |f(x)|.
  \] 
\end{defn}

The infinity norm is a valid norm for both \(B([a, b])\) and \(C([a, b])\).

\section{Topology in function spaces}
